\documentclass{article}
\usepackage[utf8]{inputenc}

\title{MIT 6.042J HW1 2005}
\author{Kelly Lin }
\date{June 2019}

\begin{document}

\maketitle

\section{Problem 1}

I tried doing this problem by cubing it, which resulted in saying something was equal to something that it definitely wasn't equal, but when I looked at the solution they had done it by squaring it and showing that it couldn't be rational based on even numbers, so I'll try to do the problem again their way. 

\section{Problem 2}
"There is a student who has emailed exactly two other people in the class, besides possibly herself."
Okay, so we know there's three students here. I'll call them s, t, and u. 
\\
$ \exists s \exists t \exists u \in S $
\\
s has emailed t and s has emailed u.
\\
$E(s,t) \wedge E(s,u)$
\\
s is a different student from u, and a different student from t, and same goes for u and t. 
\\
$s \neq u \wedge s \neq t \wedge u \neq t$
\\
Of the students that s has emailed, those students are either u, t, or themselves. So we can say "Every student o that s has emailed, o is equal to s, t, or u."
$\forall o \in S E(s, o) \rightarrow o = s \vee o = t \vee o = u$
So we can AND all of the phrases together to make the predicate. 

\section{Problem 3}

Write in predicate form. Addition, multiplication, equality, but no constants. Prove stuff first before using. This is over the domain of the natural numbers (non-negative integers).
\\
a. n is the sum of three perfect squares \\
$\exists x \exists y \exists z. x*x + y*y + z*z = n$
\\
I don't have to specify anything special on these since it's fine if x, y, and z are the same as each other. 
\\
b. x $>$ 1
\\
Since we can't use constants, we need to first define $x = 1$. 
\\
$\forall y. (y*x = y)$
\\
Now we can set things equal to one using the above definition. 
\\
$\exists y. (x > y) \wedge (y = 1)$
\\
Kind of confused on who takes the exists in the above line. 
\\
c. n is a prime number.
\\
Well, I know that it isn't possible to have two variables that multiply each other to become n. 
\\
$\neg(x*y = n)$
\\
Also have to declare it. So it's more like...
\\
$\neg (\exists x \exists y. (x > 1 \wedge y > 1 \wedge x*y = n))$
\\
And we know that n, the prime number, has to be bigger than 1. So altogether:
\\
$IS-PRIME(n) ::= (n > 1) \wedge \neg (\exists x \exists y. (x > 1 \wedge y > 1 \wedge x*y = n))$

\\
d. n is a product of two distinct primes.

e. There is no largest prime number.

f. Goldbach conjecture. Every even natural number $n > 2$ can be expressed as the sum of two primes. 

g. Bertrand's Postulate. If $n > 1$ then there is always at least one prime p such that $n < p < 2n$.

\end{document}

