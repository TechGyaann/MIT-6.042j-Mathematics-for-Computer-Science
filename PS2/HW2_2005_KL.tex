\documentclass{article}
\usepackage[utf8]{inputenc}
\usepackage{amsmath}

\title{MIT 6.042J HW2 2005}
\author{Kelly Lin }
\date{June 2019}

\begin{document}

\maketitle

\section{Problem 1}
Use induction to prove that the following inequality holds for integers $n \geq 1$.
\begin {align*}
\frac{1 * 3 * 5 ... (2n + 1)}{2 * 4 * 6 ... (2n + 2)} \geq \frac{1}{2n + 2}
\end {align*}
\\
Base case P(0) holds because result is $1/2 \geq 1/2$
\\
Restating it, P(n) is:
\begin {align*}
\frac{1 * 3 * 5 ... (2n + 1)}{2 * 4 * 6 ... (2n + 2)} \geq \frac{1}{2n + 2}
\end {align*}
And P(n+1) is:
\begin {align*}
\frac{1 * 3 * 5 ... (2n + 1)(2n + 3)}{2 * 4 * 6 ... (2n + 2)(2n + 4)} \geq \frac{1}{2(n+1) + 2}
\end {align*}
Where the right hand side simplifies to:
\begin {align*}
\frac{1 * 3 * 5 ... (2n + 1)(2n + 3)}{2 * 4 * 6 ... (2n + 2)(2n + 4)} \geq \frac{1}{2n + 4}
\end {align*}
Let's multiply both sides by 2n+3 and divide both sides by 2n+4. 
\begin {align*}
\frac{1 * 3 * 5 ... (2n + 1)*(2n + 3)}{2 * 4 * 6 ... (2n + 2)*(2n + 4)} \geq \frac{1}{2n + 2} *\frac{(2n + 3)}{(2n + 4)}
\end {align*}
We know by regular math rules that the above keeps that inequality relation. We can see that for the 2n+3 and 2n+2 on the right hand side, $\frac{2n + 3}{2n + 2}$ will always be bigger than 1. So we can say that it's contributing to making the right side having the possibility of being equal to the left hand side. So without the influence of those two, $\frac{1}{2n + 4}$ is definitely smaller than the LHS.
\begin {align*}
\frac{1 * 3 * 5 ... (2n + 1)*(2n + 3)}{2 * 4 * 6 ... (2n + 2)*(2n + 4)} > \frac{1}{2n + 4}
\end {align*}
Look, it's agreeing with P(n+1) that we stated earlier--where up there LHS is greater than or equal to RHS, and here LHS is greater than RHS. So P(n) implies P(n+1), and the inequality holds for all integers greater than or equal to n.  
\section{Problem 2}
a.
\\
\\
So we have P(n) is true for 8, 9, and 10 by showing each case of how to break those into groups of 4 or 5. But then saying that for P(n+1) that we can divide the remaining n - 3 students into groups of 4 or 5 by the assumption P(n-3) is incorrect because for n = 10, or n+1 = 11, P(n - 3) is P(7) which is not in our assumptions. And you can also see that 11 can't be divided up into groups of 4 or 5 only. \\\\
b.
\\
\\
Let P(n) be that a class with n $\geq$ 12 students can be divided into groups of 4 or 5. 
\\
We show the cases for 12 through 15:\\\\
n = 12: 4 + 4 + 4 
\\\\
n = 13: 5 + 4 + 4
\\\\
n = 14: 5 + 5 + 4
\\\\
n = 15: 5 + 5 + 5
\\\\
So how can we show that P(12),...,P(n) imply P(n+1) for all n $\geq$ 15?
\\
Assume P(12),...,P(n) are true. We have a class of n + 1 students. 
\\\\
1. Form one group of 4 students.
\\\\
2. Divide the remaining n - 3 students into groups of 4 or 5 by the assumption P(n - 3).
\\\\
3. For an n $\geq$ 15, n - 3 would be $\geq$ 12, so P(n - 3) is among our assumptions. 
\\\\
4. You can do this for any n $\geq$ 15. 
\section{Problem 3}
Stick game. \\\\
P(n) is:\\\\
If the number of sticks equals 4k + 1 for some $k \in N$, the second player has a winning strategy, otherwise, the first player has a winning strategy. 
\\
Base case P(1) is that k is equal to 0, one stick. First player is forced to take the stick and loses. 
\\\\
Assume that P(1) to P(n) are true. How can I prove that P(n+1) follows? \\\\
Proof by cases: \\\\
$n+1 = 4k$\\\\
In this case, player 1 will remove three sticks to turn the state of the sticks to $4(k - 1) + 1$ since the k drops a number and we can make up for it with the outside addition number (this made sense in my head), and then when player 2 begins her turn she has a losing strategy because it puts her in the 4k + 1 zone. 
\\\\
$n+1 = 4k + 1$\\\\
In this case, player 1 could try to remove 1 stick, but then that gives player 2 4k and that's a losing state for player 1 now. 
\\
Player 1 can try to remove 2 sticks but then that gives player 2 $4(k-1) + 3$ which is a losing state for player 1 and a winning state for player 2. 
\\
Player 1 can try to remove 3 sticks but that gives player 2 $4(k - 1) + 2$ which is a losing state for P1. 
\\\\
So in all possibilities here, P1 has the losing strategy. 
\\\\
$n+1 = 4k + 2$\\\\
In this case, player 1 can remove 1 stick to turn the state to $4k + 1$ for P2, which is a winning state for P1. 
\\\\
$n + 1 = 4k + 3$\\\\
Player 1 can remove 2 sticks to turn the state to $4k + 1$ for P2, which is a winning state for P1. 
\\\\
So in all these cases P(n + 1) stays true. 

\section {Problem 4}
a. \\\\
Honestly, I don't understand this. Let's say we have a smallest collection such that player 1 and player 2 have NO winning strategy. This collection, let's call it C, has a next state C2 after a player 1 move that cannot be a winning position for player 1 or 2 because if it did have a winning position for player 1 then player 1 was already in a winning position at the time of C. Likewise for player 2 having a winning position at state C2--they'd have had a winning position at state C. How the heck does this follow that at least one such state must have no winning strats for both players? How does it follow that such a state is a smaller collection than S? 
\\\\
I guess it's because if P1 could move so that P2 couldn't win, then in the next state when it's P2's turn there also cannot be a move P2 can do to win because then it'd mean that the previous state where P1 was was actually a P2 win and the move P1 did obviously failed to prevent the win. And if P2 is gonna lose then it means that P1 was making a move to WIN, not prevent wins in a no-win state. So this new state is ALSO a state where there's some kind of no-win move possibility, which means that it's SMALLER than the original smallest collection which is impossible, so it's a contradiction.
\\\\
b. \\\\
If the whole set A is a possible move in the game, then this means that there's only two cases after the initial state. Let's call this current special set A game GSpecial and the original game that disallows selection of set A GRegular. 
\\\\
Case 1: \\
Player 1 has a winning strategy in GRegular. In which case Player 1 plays as normal in GSpecial. Once Player 1 makes a play it's not possible anymore for Player 2 to make the Set A play because something in it has already been picked. We never said that you could play a Set A - pickedSet play, so it's done--P1 wins like if they played in a GRegular game. 
\\\\
Case 2: \\
Player 2 has a winning strategy in GRegular. In which case Player 1 just plays Set A, and now Player 1 has selected everything and P2 has nothing to pick. So by playing Set A Player 1 gets to put herself in P2's spot, and it's done. 
\\\\
In both the cases above Player 1 has winning strats, so this game is super biased. 
\end{document}

