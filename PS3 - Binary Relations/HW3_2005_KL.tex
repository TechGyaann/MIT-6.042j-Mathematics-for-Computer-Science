\documentclass{article}
\usepackage[utf8]{inputenc}
\usepackage{amsmath}

\title{MIT 6.042J HW3 2010}
\author{ Kelly Lin }
\date{June 2019}

\begin{document}

\maketitle

\section{Problem 1}
On the set {0, 1}, list out the binary relations. 
\\
Empty set
\\
1. (0,0)
\\
2. (1,1)
\\
3. (0,1)
\\
4. (1,0)
\\
5. (0,0)(1,1)
\\
6. (0,0)(1,0)
\\
7. (0,0)(0,1)
\\
8. (1,1)(1,0)
\\
9. (1,1)(0,1)
\\
10. (1,0)(0,1)
\\
11. (0,0)(1,0)(0,1)
\\
12. (0,0)(1,1)(0,1)
\\
13. (0,0)(1,1)(1,0)
\\
14. (1,0)(0,1)(1,1)
\\
15. (0,0)(1,0)(0,1)(1,1)
\\\\
b. Which are weak partial orders? Strict partial orders? Equivalence relations?
\\\\
Empty set is symmetric. 

I don't know if this is the best way to think of it, but reflexive is when all numbers included in the relation are connected to themselves. Transitive is when a number A visits another number B and number B visits a number C (which could be A), you also have the number A visiting C listed. Symmetric is when a number A going to another number B also then has B going to A. 

So the reflexive ones are: \\
5, 12, 13, 15
\\ The reason why none of the single group ones are reflexive is because reflexivity applies over the set, and our set has both 0 and 1 so it needs 0,0 and 1,1 to be reflexive. 
\\\\
Transitive ones are:\\
Empty set, 1, 2, 3, 4, 5, 6, 7, 8, 9, 12, 13, 15, 
\\\\
Symmetric ones are:\\
Empty set, 1, 2, 5, 10, 11, 14, 15
\\\\
Antisymmetric confuses me. The definition in the text says that it's if $aRb \rightarrow \neg (bRa)$, but that is confusing me. Online I see someone saying it's antisymmetric unless there's some a and b for which aRb and bRa yet nevertheless a is not equal to b. So I'm guessing that the text definition intends that a and b are unique and not equal. So if there's a unique a and there's a unique b and they have that relation between them such that aRb, then it follows that it doesn't go back the way around to bRa. So if there's some a and b where b IS a, and they have the relation of aRb and bRa, that's fine and still antisymmetric. 
\\
\\
Antisymmetric ones are: \\
Empty set, 1, 2, 3, 4, 5, 6, 7, 8, 9, 12, 13
\\\\
Okay, now that that's out of the way, which are weak partial orders? Weak partial orders are transitive, antisymmetric, and reflexive. So that's: \\
5, 12, and 13
\\\\
The strict partial orders are those that are transitive, antisymmetric, and irreflexive. \\
Empty set, 3, 4. Irreflexive is different from nonreflexive--everything in it needs to not be reflexive individually for it to be irreflexive, so 1 and 2 are out.
\\\\
Equivalence relations are transitive, symmetric, and reflexive. \\\\
5 and 15.

\section{Problem 2}
a. Partially order the power set P({1,2,...,n} by subset relation. Describe a maximum length chain and explain why there cannot be a longer chain that it. 
\\\\
The maximum length chain would be one where we have Empty Set, {1}, {1,2}, {1,2,3}, {1,2,3,4}, ... And onwards until it is of a total length n + 1 (because of the empty set). \\
There can't be a chain longer than this because we've gone through the entire power set. Can't have repeats, and if we had duplicate sized sets made up of different numbers, they wouldn't be subsets of each other and would break the chain. 
\\\\
b. Describe a topological sort and explain why your sort is correct.
\\
A topological sort would be starting from the empty set (since it is a subset of every set), then all sets that are length 1, like {1} or {2}, in any order. Then all sets that are length 2, and so on until a set of all the numbers up to n. That last one would contain all the ones before it. 
\end{document}

