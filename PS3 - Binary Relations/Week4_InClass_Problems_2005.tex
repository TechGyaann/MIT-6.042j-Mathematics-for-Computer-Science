\documentclass{article}
\usepackage[utf8]{inputenc}
\usepackage{amsmath}

\title{2005 In-Class Problems Week 4}
\author{ Kelly Lin }
\date{July 2019}

\begin{document}

\maketitle

\section{Problem 1}
Problem 1. In each case, say whether or not R is a equivalence relation on A. If it is an equivalence
relation, what are the equivalence classes and how many equivalence classes are there?
\\\\
(a) R ::= {$(x, y) \in W \times W $| the words x and y start with the same letter} where W is the set of all words in the 2001 edition of the Oxford English dictionary.
\\\\
Yes, R is an equivalence relation. It's reflexive because a word starting with the same letter indeed is related in this way to itself. Likewise it's transitive and symmetric. The equivalence classes are the letters of the alphabet, of which there are 26 in the English alphabet. So 26 equivalence classes. 
\\\\
(b) R ::= {$(x, y) \in W \times W |$ the words x and y have at least one letter in common}.
\\\\
This is reflexive and symmetric but it is not transitive. A word like "tap" and "soothe" are related, and "soothe" and "son" are related, but "tap" and "son" are not. So it is not an equivalence relation.
\\\\
(c) R = {$(x, y) \in W \times W$ and the word x comes before the word y alphabetically}.
\\\\
This is transitive. But it isn't reflexive since a word doesn't come before itself. It also isn't symmetric. It's actually antisymmetric, since if xRy and x is not equal to y, it is impossible for yRx. So it isn't an equivalence relation. However, it's a partial order.
\\\\
(d) R = ${(x, y) \in R \times R$ and abs( x ) less than or equal to abs( y )}.
\\\\
This is reflexive because of the equality part of the operator. It is not symmetric. It is transitive. It is not antisymmetric. If xRy here and x is not equal to y, it is still possible for yRx if they're opposing signs. 
\\\\
(e) R = {$(x, y) \in B \times B$, where B is the set of all bit strings and x and y have the same number of
1s.}
\\\\
This is an equivalence relation. It's reflexive, transitive, and symmetric. The equivalence class is how many ones there are. So there are as many classes as there are possible to be ones in the bitstring. 

\section{Problem 2}
False claim: suppose R is a relation on A. If R is symmetric and transitive, then R is reflexive. \\\\
a. Give a counter example to the claim.
\\
Riffing off of HW3, let's say we have a set that contains {0,1,2}. If we have a relation on the set that is {(0,0),(1,1),(0,1),(1,0)}, this relation is symmetric and transitive but it isn't reflexive because 2 does not relate to itself.
\\\\
b. Find the flaw in the following proof.
\\
In the proof provided, it says let x be an arbitrary element of A. Then let y be any element of A such that xRy. In our previous example though, what if x was 2? Then there is no element y where xRy, because 2 is not related to anything at all. So that's the issue with the proof--that there may not be an element y such that xRy. 
\\\\
\section{Problem 3}
Describe a function g that verifies each relation is a partial order. Is it a total order?
\\
a. The relation $<$ on all real numbers, R.
\\\\
If we define a function g that takes a real number r and returns the set of all real numbers that is less than r, this satisfies the a1 R a2 iff g(a1) $\subset$ g(a2) for all a1 $\neq$ a2 $\in$ A.
\begin {align*}
g(r) ::== {t \in R | t < r}
\end {align*}
This works because we're getting back the set of all real numbers less than the r we gave as an input, and then r1 < r2 iff g(r1) $\subset$ g(r2). If a set of all these numbers less than r1 is a subset of the set of all these numbers less than r2, then you know r1 is less than r2. 
\\\\
This is a total order, because every item is comparable. There's nothing that can't be compared to another.
\\\\
b. The superset relation $\supseteq$ on the power set P(B) for a set B. 
\\\\
We can define the function g as the set of all elements with the input set removed. 
\begin {align*}
g(a) ::== B - {a}
\end {align*}
For any a in the powerset of B where the a1 and a2 are not equal, a1 is an inclusive superset of a2 iff a1 is a proper superset of a2, since a1 and a2 are not equal. B minus a1 is then a subset of B minus a2. And that's what our function is, g(a1) $\subset$ g(a2). So if our function holds then a1 R a2. 
\\\\
This isn't a total order though, because a set could be incomparable to another set. Like if B is {1,2,3}, we could have one set be {1,3} and another be {2,3} and neither are subsets or supersets of the other. 
\\\\
c. The "divides" relation on natural numbers.
\\\\
We can make g(a) be the function that returns the set of natural numbers that divide a. This isn't a total order though, since there are natural numbers that don't divide the other. They aren't comparable. In order for it to be a total order, it'd mean that for any two natural numbers, one divided the other. 
\\\\
\section{Problem 4}
a. Describe another couple of minimal sets of pairs that determine the relation.
\\
Some examples:
\\
{66,31,54,75}\\
{66,13,54,75}\\
{66,13,47,75}\\
\\\\
b. Here are the pairs left after some unnecessary pairs have been removed from an equivalence relation E. What is the domain of E? What are the equivalence classes of E?
\\
Domain of E goes from 0 to 9 and then a to c. So {0123456789abc}.\\ Looks like the equivalence classes are {024},{1579a},{68},{b},{c3}. 
\\\\
c. On a domain of n elements, what is the smallest number of pairs that could determine an equivalence relation?
\\
Don't really understand. I guess it's that it's the ceiling round of n/2 because if I have for example 2 elements {1,2} I can just have one pair, (12). 3 elements can have two pairs, (12) and (33), so the block size is size 2 and one size 1 for odd numbers. Likewise, for 4 elements I have 2 pairs, (12)(34) and for 5 elements I have 3 pairs, two that are size 2 blocks and one a size 1 block (12)(34)(55). 
\\\\
d. Suppose you have an equivalence relation on a domain of size n with k equivalence classes,
with no classes of just one element. Then every minimal set of pairs has the same size. What is
that size? Explain.
\\\\
Also don't really understand this. I've got size n elements. k equivalence classes. The classes at minimum have 2 elements. And every set of pairs has the same size. Solution says it takes n - k pairs, but I don't get why? 

\end{document}



